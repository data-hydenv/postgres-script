%-------------------------------------------------------
% LaTeX 101
% bold - \textbf{bold text}
% italic - \textit{italic text}
%
% bullet points:
% \begin{itemize}
%   \item first point
%   \item second point
% \end{itemize}
%
% Code listings
% \begin{lstlisting}[language=sql]
% SELECT * FROM metadata
% \end{lstlisting}
% you can also change the language to Python or R
%
%-------------------------------------------------------
\section{Normalization}

The overall objective of normalization is to enhance the data model so that it \blank.
This is achieved by a number of so called normal forms, which are hierarchical.

\subsection{first normal form}

A data model is in the first normal, when \blank.

An easy example is: \blank.

\subsection{second normal form}

A data model is in the second normal form, when it is in the \blank normal form and \blank.

For the HOBO data, this is applied at: \blank. 

\subsection{third normal form}

A data model in in the third normal form, when it is in the \blank normal form \blank.

For the HOBO data, this could be applied like: \blank.